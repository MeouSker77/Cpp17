%! Author = wys
%! Date = 2020/9/23

% Preamble
\documentclass[a4paper, 12pt, AutoFakeBold, AutoFakeSlant]{article}

% Packages
\usepackage{xeCJK}
\usepackage{indentfirst}
\usepackage{listings}
\usepackage{enumitem}
\usepackage{graphicx}
\usepackage{amsmath}
\usepackage[colorlinks, linkcolor=black]{hyperref}
\usepackage{xcolor}
\usepackage[font={it}]{caption}
\usepackage[numbered]{bookmark}

\setCJKmainfont{宋体}
\setCJKmonofont{宋体}
\setmainfont{Times New Roman}
\setmonofont{Noto Sans Mono}
\linespread{1.25}
\CJKsetecglue{\,}
\numberwithin{figure}{section}

\lstset{
basicstyle = \ttfamily\footnotesize,
breakatwhitespace = false,
breaklines = true,
commentstyle = \color{gray}\itshape,
extendedchars = false,
keepspaces=true,
keywordstyle=\color{blue}, % keyword style
language = C++,            % the language of code
otherkeywords={string},
rulecolor=\color{black},
showspaces=false,
showstringspaces=false,
showtabs=false,
stringstyle=\color{magenta},        % string literal style
tabsize=4,
}

%\includeonly{ch8}

% Document
\begin{document}
    \tableofcontents
    \clearpage
    \setcounter{page}{1}


    \part{基本语言特性}\label{part1}
    这一部分介绍了C++17中新的核心语言特性,但不包括那些专为泛型编程(即template)设计的特性。
    这些新增的特性对于应用程序员的日常编程非常有用,因此每一个使用C++17的C++程序员都应该了解它们。

    专为模板编程设计的新的核心语言特性在\autoref{part2}中介绍。

    \section{结构化绑定}\label{ch1}
结构化绑定允许你用一个对象的元素或成员同时实例化多个实体。
例如,假设你定义了一个有两个不同成员的结构体:
\begin{lstlisting}
    struct MyStruct {
        int i = 0;
        std::string s;
    };

    MyStruct ms;
\end{lstlisting}
你可以通过如下的声明直接把该结构体的两个成员绑定到新的变量名:
\begin{lstlisting}
    auto [u, v] = ms;
\end{lstlisting}
这里,变量\texttt{u}和\texttt{v}的声明方式称为\emph{结构化绑定}。
某种程度上可以说它们解构了用来初始化的对象(有些观点称它们为\emph{解构声明})。

如下的每一种声明方式都是支持的:
\begin{lstlisting}
    auto [u2, v2] {ms};
    auto [u3, v3] (ms);
\end{lstlisting}
结构化绑定对于返回结构体或者数组的函数来说非常有用。例如,考虑一个返回结构体的函数:
\begin{lstlisting}
    MyStruct getStruct() {
        return MyStruct{42, "hello"};
    }
\end{lstlisting}
你可以直接把返回的数据成员赋值给两个新的局部变量:
\begin{lstlisting}
    auto [id, val] = getStruct();   // id和val分别是返回结构体中的i和s成员
\end{lstlisting}
这里,\texttt{id}和\texttt{val}分别是返回结构体中的\texttt{i}和\texttt{s}成员。
它们的类型分别对应\texttt{int}和\texttt{std::string},可以被当作两个不同的对象来使用:
\begin{lstlisting}
    if (id > 30) {
        std::cout << val;
    }
\end{lstlisting}
这么做的好处是可以直接访问成员,
另外通过把值绑定到能体现语义的变量名可以使代码的可读性更强\footnote{感谢Zachary Turner指出这一点}。

下面的代码演示了使用结构化绑定带来的显著改进。
在不使用结构化绑定的情况下遍历\texttt{std::map<>}的元素需要这么写:
\begin{lstlisting}
    for (const auto& elem : mymap) {
        std::cout << elem.first << ": " << elem.second << '\n';
    }
\end{lstlisting}
元素的类型是键和值组成的\texttt{std::pair}类型,
\texttt{std::pair}的成员分别是\texttt{first}和\texttt{second},
上边的例子中必须使用成员的名字来访问键和值。通过使用结构化绑定,代码的可读性大大提升:
\begin{lstlisting}
    for (const auto& [key, val] : mymap) {
        std::cout << key << ": " << val << '\n';
    }
\end{lstlisting}
上面的例子中我们可以使用准确体现语义的变量名直接访问每一个元素。

\subsection{细说结构化绑定}
为了理解结构化绑定,必须意识到这里面其实有一个隐藏的匿名对象。
结构化绑定时新引入的局部变量名其实都指向这个匿名对象的成员/元素。

\subsubsection*{绑定到一个匿名实体}
如下代码的精确行为:
\begin{lstlisting}
    auto [u, v] = ms;
\end{lstlisting}
其实等价于我们用\texttt{ms}初始化了一个新的实体\texttt{e},
并且让结构化绑定中的\texttt{u}和\texttt{v}变成了\texttt{e}的成员的别名,类似于如下定义:
\begin{lstlisting}
    auto e = ms;
    aliasname u = e.i;
    aliasname v = e.s;
\end{lstlisting}
这意味着\texttt{u}和\texttt{v}仅仅是\texttt{ms}的一份本地拷贝的成员的别名。
然而,我们没有为\texttt{e}声明一个名称,因此我们不能直接访问这个匿名对象。
注意\texttt{u}和\texttt{v}并不是\texttt{e.i}和\texttt{e.s}的引用(而是它们的别名)。
\texttt{decltype(u)}的结果是成员\texttt{i}的类型,
\texttt{declytpe(v)}的结果是成员\texttt{s}的类型。
因此:
\begin{lstlisting}
    std::cout << u << ' ' << v << '\n';
\end{lstlisting}
会打印出\texttt{e.i}和\texttt{e.s}(分别是\texttt{ms.i}和\texttt{ms.s}的拷贝)。

\texttt{e}的生命周期和结构化绑定的生命周期相同,当结构化绑定离开作用域时\texttt{e}也会被自动销毁。
另外,除非使用了引用,否则修改结构化绑定的变量并不会影响被绑定的变量:
\begin{lstlisting}
    MyStruct ms{42, "hello"};
    auto [u, v] = ms;
    ms.i = 77;
    std::cout << u;     // 打印出42
    u = 99;
    std::cout << ms.i;  // 打印出77
\end{lstlisting}
在这个例子中\texttt{u}和\texttt{ms.i}有不同的内存地址。

当使用结构化绑定来绑定返回值时,规则是相同的。如下初始化
\begin{lstlisting}
    auto [u, v] = getStruct();
\end{lstlisting}
的行为等价于我们用\texttt{getStruct()}的返回值初始化了一个新的实体\texttt{e},
之后结构化绑定的变量\texttt{u}和\texttt{v}变成了\texttt{e}的两个成员的别名,类似于如下定义:
\begin{lstlisting}
    auto e = getStruct();
    aliasname u = e.i;
    aliasname v = e.s;
\end{lstlisting}
也就是说,结构化绑定绑定到了一个新的实体\texttt{e}上,而不是直接绑定到了返回值上。
匿名实体\texttt{e}同样遵循通常的内存对齐规则,结构化绑定的每一个变量都会根据相应成员的类型进行对齐。

\subsubsection*{使用修饰符}
我们可以在结构化绑定中使用修饰符,例如\texttt{const}和引用,这些修饰符会作用在匿名实体\texttt{e}上。
通常情况下,作用在匿名实体上和作用在结构化绑定的变量上的效果是一样的,但有些时候又是不同的(见下文)。

例如,我们可以把声明一个结构化绑定声明为\texttt{const}引用:
\begin{lstlisting}
    const auto& [u, v] = ms;    // 引用,因此u/v指向ms.i/ms.s
\end{lstlisting}
这里,匿名实体被声明为\texttt{const}引用,
而\texttt{u}和\texttt{v}分别是这个引用的成员\texttt{i}和\texttt{s}的别名。
因此,对\texttt{ms}的成员的修改会影响到\texttt{u}和\texttt{v}的值:
\begin{lstlisting}
    ms.i = 77;          // 影响u的值
    std::cout << u;     // 打印出77
\end{lstlisting}
如果声明为非\texttt{const}引用,你甚至可以修改对象的成员:
\begin{lstlisting}
    MyStruct ms{42, "hello"};
    auto& [u, v] = ms;      // 被初始化的实体是ms的引用
    ms.i = 77;              // 影响到u的值
    std::cout << u;         // 打印出77
    u = 99;                 // 修改了ms.i
    std::cout << ms.i;      // 打印出99
\end{lstlisting}
如果一个结构化绑定是引用类型,而且是对一个临时对象的引用,那么和往常一样,
临时对象的生命周期会被延长到结构化绑定的生命周期:
\begin{lstlisting}
    MyStruct getStruct();
    ...
    const auto& [a, b] = getStruct();
    std::cout << "a: " << a << '\n';    // OK
\end{lstlisting}

\subsubsection*{修饰符并不是作用在结构化绑定引入的变量上}
修饰符会作用在新的匿名实体上,而不是结构化绑定引入的新的变量名上。事实上,如下代码中:
\begin{lstlisting}
    const auto& [u, v] = ms;    // 引用,因此u/v指向ms.i/ms.s
\end{lstlisting}
无论是\texttt{u}还是\texttt{v}都不是引用, 只有匿名实体\texttt{e}是一个引用。
\texttt{u}和\texttt{v}分别是\texttt{ms}对应的成员的类型,
只不过变成了\texttt{const}的。
根据我们的推导,\texttt{decltype(u)}是\texttt{const int},
\texttt{decltype(v)}是\texttt{const std::string}。

当声明对齐时也是类似:
\begin{lstlisting}
    alignas(16) auto [u, v] = ms;   // 对齐匿名实体,而不是v
\end{lstlisting}
这里,我们对齐了匿名实体而不是\texttt{u}和\texttt{v}。
这意味着\texttt{u}作为第一个成员会按照16字节对齐,但\texttt{v}不会。

因此,即使使用了\texttt{auto}结构化绑定也不会发生类型退化(\emph{decay})
\footnote{术语\emph{decay}是指当参数按值传递时发生的类型转换,
例如原生数组会转换为指针,顶层修饰符例如\texttt{const}和引用会被忽略}。
例如,如果我们有一个原生数组组成的结构体:
\begin{lstlisting}
    struct S {
        const char x[6];
        const char y[3];
    };
\end{lstlisting}
那么如下声明之后:
\begin{lstlisting}
    S s1{};
    auto [a, b] = s1;    // a和b的类型是结构体成员的精确类型
\end{lstlisting}
这里\texttt{a}的类型仍然是\texttt{char[6]}。
再次强调,\texttt{auto}关键字应用在匿名实体上,这里匿名实体整体并不会发生类型退化。
这和用\texttt{auto}初始化新对象不同,如下代码中会发生类型退化:
\begin{lstlisting}
    auto a2 = a;    // a2的类型是a的退化类型
\end{lstlisting}

\subsubsection*{move语义}
move语义也遵循之前介绍的规则,如下声明:
\begin{lstlisting}
    MyStruct ms = { 42, "Jim" };
    auto&& [v, n] = std::move(ms);     // 匿名实体是ms的右值引用
\end{lstlisting}
这里\texttt{v}和\texttt{n}指向的匿名实体是\texttt{ms}的右值引用。
同时\texttt{ms}的值仍然保持不变:
\begin{lstlisting}
    std::cout << "ms.s: " << ms.s << '\n';  // 打印出"Jim"
\end{lstlisting}
然而,你可以对指向\texttt{ms.s}的\texttt{n}进行移动赋值:
\begin{lstlisting}
    std::string s = std::move(n);   // 把ms.s移动到s
    std::cout << "ms.s: " << ms.s << '\n';  // 打印出未定义的值
    std::cout << "n:    " << n << '\n';     // 打印出未定义的值
    std::cout << "s:    " << s << '\n';     // 打印出"Jim"
\end{lstlisting}
像通常一样值被移动走的对象处于一个值未定义但却有效的状态。因此打印它们的值是没有问题的,
但不要对打印出的值做任何假设\footnote{对于\texttt{string}来说,
值被移动走之后一般是处于空字符串的状态,但并不保证这一点}。

上面的例子和直接用\texttt{ms}被移动走的值进行结构化绑定有些不同:
\begin{lstlisting}
    MyStruct ms = {42, "Jim" };
    auto [v, n] = std::move(ms);    // 新的匿名实体持有从ms处移动走的值
\end{lstlisting}
这里新的匿名实体是用\texttt{ms}被移动走的值来初始化的。因此,\texttt{ms}已经失去了值:
\begin{lstlisting}
    std::cout << "ms.s: " << ms.s << '\n';  // 打印出未定义的值
    std::cout << "n:    " << n << '\n';     // 打印出"Jim"
\end{lstlisting}
你可以继续用\texttt{n}进行移动赋值或者给\texttt{n}赋予新值,但已经不会再影响到\texttt{ms.s}了:
\begin{lstlisting}
    std::string s = std::move(n);   // 把n移动到s
    n = "Lara";
    std::cout << "ms.s: " << ms.s << '\n';  // 打印出未定义的值
    std::cout << "n:    " << n << '\n';     // 打印出"Lara"
    std::cout << "s:    " << s << '\n';     // 打印出"Jim"
\end{lstlisting}

\subsection{结构化绑定的适用场景}
原则上讲,结构化绑定适用于所有只有\texttt{public}数据成员的结构体、
C风格数组和类似元组(tuple-like)的对象:
\begin{itemize}[leftmargin=*]
    \item 对于所有非静态数据成员都是\texttt{public}的\textbf{结构体和类},
    你可以把每一个成员绑定到一个新的变量名上。
    \item 对于\textbf{原生数组},你可以把数组的每一个元素绑定到新的变量名上。
    \item 对于任何类型,你可以使用\textbf{tuple-like API}来绑定新的名称,
    无论这套API是如何定义“元素”的。对于一个类型\emph{type}这套API需要如下的组件:
    \begin{itemize}[leftmargin=*]
        \item \texttt{std::tuple\_size<type>::value}要返回元素的数量。
        \item \texttt{std::tuple\_element<idx, type>::type}
        要返回第\texttt{idx}个元素的类型。
        \item 一个全局或成员函数\texttt{get<idx>()}要返回第\texttt{idx}个元素的值。
    \end{itemize}
    标准库类型\texttt{std::pair<>}、\texttt{std::tuple<>}、\texttt{std::array<>}
    就是提供了这些API的示例。
\end{itemize}
如果结构体和类提供了tuple-like API,那么将会使用这些API进行绑定,而不是直接绑定数据成员。

在任何情况下,结构化绑定中声明的变量名的数量必须和元素或数据成员的数量相同。
你不能跳过某个元素,也不能重复使用变量名。然而,你可以使用非常短的名称例如\texttt{'\_'}
(有的程序员喜欢这个名字,有的讨厌它,但注意全局命名空间不允许使用它),
但这个名字在同一个作用域只能使用一次:
\begin{lstlisting}
    auto [_, val1] = getStruct();   // OK
    auto [_, val2] = getStruct();   // ERROR:变量名_已经被使用过
\end{lstlisting}
目前还不支持嵌套化的结构化绑定。

下一小节将详细讨论结构化绑定的使用。

\subsubsection{结构体和类}
上面几节里已经介绍了对只有\texttt{public}成员的结构体和类使用结构化绑定的方法,
一个典型的应用是直接对包含多个数据的返回值使用结构化绑定。然而有一些边缘情况需要注意。

注意要使用结构化绑定需要继承时遵循一定的规则。所有的非静态数据成员必须在同一个类中定义
(也就是说,这些成员要么是全部直接来自于最终的类,要么是全部来自同一个父类):
\begin{lstlisting}
    struct B {
        int a = 1;
        int b = 2;
    };

    struct D1 : B {
    };
    auto [x, y] = D1{};     // OK

    struct D2 : B {
        int c = 3;
    };
    auto [i, j, k] = D2{};  // 编译期ERROR
\end{lstlisting}
注意只有当\texttt{public}成员的顺序保证是固定的时候你才应该使用结构化绑定。
否则如果\texttt{B}中的\texttt{int a}和\texttt{int b}的顺序发生了变化,
\texttt{x}和\texttt{y}的值也会随之变化。为了保证固定的顺序,
C++17为一些标准库结构体(例如\texttt{insert\_return\_type})定义了成员顺序。

联合还不支持使用结构化绑定。

\subsubsection{原生数组}
下面的代码用C风格数组的两个元素初始化了\texttt{x}和\texttt{y}:
\begin{lstlisting}
    int arr[] = { 47, 11 };
    auto [x, y] = arr;  // x和y是arr中的int元素的拷贝
    auto [z] = arr;     // ERROR:元素的数量不匹配
\end{lstlisting}
注意这是C++中少数几种原生数组会按值拷贝的场景之一。

只有当数组的长度已知时才可以使用结构化绑定。
对于传递进入的数组参数不能使用结构化绑定,因为数组会\emph{退化(decay)}为相应的指针类型。

注意C++允许通过引用来返回带有大小信息的数组,结构化绑定可以应用于返回这种数组的函数:
\begin{lstlisting}
    auto getArr() -> int(&)[2];     // getArr()返回一个原生int数组的引用
    ...
    auto [x, y] = getArr();     // x和y是返回的数组中的int元素的拷贝
\end{lstlisting}
你也可以对\texttt{std::array}使用结构化绑定,这是通过下一节要讲述的tuple-like API来实现的。

\subsubsection{\texttt{std::pair}, \texttt{std::tuple}和\texttt{std::array}}
结构化绑定的机制是可拓展的,你可以为任何类型添加对绑定的支持。
标准库中就为\texttt{std::pair<>}、\texttt{std::tuple<>}、
\texttt{std::array<>}添加了支持。

\subsubsection*{\texttt{std::array}}
例如,下面的代码为\texttt{getArray()}返回的\texttt{std::array<>}中的四个元素绑定了
新的变量名\texttt{a},\texttt{b},\texttt{c},\texttt{d}:
\begin{lstlisting}
    std::array<int, 4> getArray();
    ...
    auto [a, b, c, d] = getArray(); // a,b,c,d是返回值的拷贝中的四个元素的别名
\end{lstlisting}
这里\texttt{a},\texttt{b},\texttt{c},\texttt{d}被绑定到\texttt{getArray()}返回的
\texttt{std::array}的元素上。

使用非临时变量的\texttt{non-const}引用进行绑定,还可以进行修改操作。例如:
\begin{lstlisting}
    std::array<int, 4> stdarr { 1, 2, 3, 4 };
    ...
    auto& [a, b, c, d] = stdarr;
    a += 10;    // OK:修改了stdarr[0]

    const auto& [e, f, g, h] = stdarr;
    e += 10;    // ERROR:引用指向常量对象

    auto&& [i, j, k, l] = stdarr;
    i += 10;    // OK:修改了stdarr[0]

    auto [m, n, o, p] = stdarr;
    m += 10;    // OK:但是修改的是stdarr[0]的拷贝
\end{lstlisting}
然而像往常一样,我们不能用临时对象(prvalue)初始化一个 \texttt{non-const}引用:
\begin{lstlisting}
    auto& [a, b, c, d] = getArray();    // ERROR
\end{lstlisting}

\subsubsection*{\texttt{std::tuple}}
下面的代码将\texttt{a},\texttt{b},\texttt{c}初始化为\texttt{getTuple()}返回的
\texttt{std::tuple<>}的拷贝的三个元素的别名:
\begin{lstlisting}
    std::tuple<char, float, std::string> getTuple();
    ...
    auto [a, b, c] = getTuple();    // a,b,c的类型和值与返回的tuple中相应的成员相同
\end{lstlisting}
其中\texttt{a}的类型是\texttt{char},\texttt{b}的类型是\texttt{float},
\texttt{c}的类型是\texttt{std::string}。

\subsubsection*{\texttt{std::pair}}
作为另一个例子,考虑如下对关联/无序容器的\texttt{insert()}成员的返回值进行处理的代码:
\begin{lstlisting}
    std::map<std::string, int> coll;
    auto ret = coll.insert({"new", 42});
    if (!ret.second) {
        // 如果插入失败,使用ret.first处理错误
        ...
    }
\end{lstlisting}
通过使用结构化绑定,而不是使用\texttt{std::pair<>}的\texttt{first}和
\texttt{second}成员,代码的可读性大大增强:
\begin{lstlisting}
    auto [pos, ok] = coll.insert({"new", 42});
    if (!ok) {
        // 如果插入失败,用pos处理错误
        ...
    }
\end{lstlisting}
注意在这种场景中,C++17中提供了一种使用\hyperref[ch2]{带初始化的\texttt{if}语句}
来进行改进的方法。

\subsubsection*{为\texttt{pair}和\texttt{tuple}的结构化绑定赋予新值}
在声明了一个结构化绑定之后,你通常不能一起修改所有绑定的变量。
因为结构化绑定只能一起声明但不能一起使用。然而你可以使用\texttt{std::tie()}
把值一起赋给所有变量。例如:
\begin{lstlisting}
    std::tuple<char, float, std::string> getTuple();
    ...
    auto [a, b, c] = getTuple();    // a,b,c的类型和值与返回的tuple相同
    ...
    std::tie(a, b, c) = getTuple(); // a,b,c的值变为新返回的tuple的值
\end{lstlisting}
这种方法可以被用来处理返回多个值的循环,例如\hyperref[ch24.1.2]{在循环中使用搜索器}:
\begin{lstlisting}
    std::boyer_moore_searcher bmsearch{sub.begin(), sub.end()};
    for (auto [beg, end] = bmsearch(text.begin(), text.end());
         beg != text.end();
         std::tie(beg, end) = bmsearch(end, text.end())) {
        ...
    }
\end{lstlisting}

\subsection{为结构化绑定提供Tuple-Like API}
你可以通过\emph{tuple-like API}为任何类型添加对结构化绑定的支持,
就像标准库中为\texttt{std::pair<>}、\texttt{std::tuple}、\texttt{std::array<>}做的一样:

\subsubsection*{支持只读结构化绑定}
下面的例子演示了怎么为一个类型\texttt{Customer}添加结构化绑定支持,
类的定义如下:
\begin{lstlisting}[frame=single, title=lang/customer1.hpp]
    #include <string>
    #include <utility>  // for std::move()

    class Customer {
      private:
        std::string first;
        std::string last;
        long val;
      public:
        Customer (std::string f, std::string l, long v)
            : first{std::move(f)}, last{std::move(l)}, val{v} {
        }
        std::string getFirst() const {
            return first;
        }
        std::string getLast() const {
            return last;
        }
        long getValue() const {
            return val;
        }
    };
\end{lstlisting}
我们可以用如下代码添加tuple-like API:
\begin{lstlisting}[frame=single, title=lang/structbind1.hpp]
    #include "customer1.hpp"
    #include <utility>  // for tuple-like API

    // 为类Customer提供tulpe-like API:
    template<>
    struct std::tuple_size<Customer> {
        static constexpr int value = 3; // 有三个属性
    };

    template<>
    struct std::tuple_element<2, Customer> {
        using type = long;  // 最后一个属性的类型是long
    };
    template<std::size_t Idx>
    struct std::tuple_element<Idx, Customer> {
        using type = std::string;   // 其他的属性都是string
    };

    // 定义特化的getter:
    template<std::size_t> auto get(const Customer& c);
    template<> auto get<0>(const Customer& c) { return c.getFirst(); }
    template<> auto get<1>(const Customer& c) { return c.getLast(); }
    template<> auto get<2>(const Customer& c) { return c.getValue(); }
\end{lstlisting}
这里,我们为顾客的三个属性定义了tuple-like API,并映射到三个getter:
\begin{itemize}[leftmargin=*]
    \item 顾客的姓是\texttt{std::string}类型
    \item 顾客的名是\texttt{std::string}类型
    \item 顾客的消费金额是\texttt{long}类型
\end{itemize}
属性的数量被定义为\texttt{std::tuple\_size}模板函数对类\texttt{Customer}的特化版本:
\begin{lstlisting}
    template<>
    struct std::tuple_size<Customer> {
        static constexpr int value = 3; // 我们有3个属性
    };
\end{lstlisting}
属性的类型被定义为\texttt{std::tuple\_element}的特化版本:
\begin{lstlisting}
    template<>
    struct std::tuple_element<2, Customer> {
        using type = long;  // 最后一个属性是long类型
    };
    template<std::size_t Idx>
    struct std::tuple_element<Idx, Customer> {
        using type = std::string;   // 其他的属性是string
    };
\end{lstlisting}
第三个属性是\texttt{long},被定义为\texttt{Idx}为2时的完全特化版本。
其他的属性类型都是\texttt{std::string},被定义为部分特化版本(优先级比全特化版本低)。
这里的类型就是结构化绑定时\texttt{decltype}返回的类型。

最后在和类\texttt{Customer}相同的命名空间定义了模板函数\texttt{get<>()}的
重载版本作为getter\footnote{C++17标准也允许把\texttt{get<>()}函数定义为成员函数,
但这可能只是一个疏忽,因此不应该这么用}:
\begin{lstlisting}
    template<std::size_t> auto get(const Customer& c);
    template<> auto get<0>(const Customer& c) { return c.getFirst(); }
    template<> auto get<1>(const Customer& c) { return c.getLast(); }
    template<> auto get<2>(const Customer& c) { return c.getValue(); }
\end{lstlisting}
在这种情况下,我们有一个主函数模板的声明和针对所有情况的全特化版本。

注意函数模板的全特化版本必须使用和声明时相同的类型(包括返回值类型都必须完全相同)。
这是因为我们只是提供特化版本的实现,而不是新的声明。下面的代码将不能通过编译:
\begin{lstlisting}
    template<std::size_t> auto get(const Customer& c);
    template<> std::string get<0>(const Customer& c) { return c.getFirst(); }
    template<> std::string get<1>(const Customer& c) { return c.getLast(); }
    template<> long get<2>(const Customer& c) { return c.getValue(); }
\end{lstlisting}
通过使用新的\hyperref[ch10]{编译期\texttt{if}语句特性},我们可以把\texttt{get<>()}函数的实现
合并到一个函数里:
\begin{lstlisting}
    template<std::size_t I> auto get(const Customer& c) {
        static_assert(I < 3);
        if constexpr (I == 0) {
            return c.getFirst();
        }
        else if constexpr (I == 1) {
            return c.getLast();
        }
        else {  // I == 2
            return c.getValue();
        }
    }
\end{lstlisting}
有了这个API,我们就可以为类型\texttt{Customer}使用结构化绑定:
\begin{lstlisting}[frame=single, title=lang/structbind1.cpp]
    #include "structbind1.hpp"
    #include <iostream>

    int main()
    {
        Customer c{"Tim", "Starr", 42};

        auto [f, l, v] = c;

        std::cout << "f/l/v:    " << f << ' '
                  << l << ' ' << v << '\n';
        // 修改结构化绑定的变量
        std::string s{std::move(f)};
        l = "Waters";
        v += 10;
        std::cout << "f/l/v:    " << f << ' '
                  << l << ' ' << v << '\n';
        std::cout << "c:        " << c.getFirst() << ' '
                  << c.getLast() << ' ' << c.getValue() << '\n';
        std::cout << "s:        " << s << '\n';
    }
\end{lstlisting}
如下初始化之后:
\begin{lstlisting}
    auto [f, l, v] = c;
\end{lstlisting}
像之前的例子一样,\texttt{Customer c}被拷贝到一个匿名实体。
当结构化绑定离开作用域时匿名实体也被销毁。

另外,对于每一个绑定\texttt{f}、\texttt{l}、\texttt{v},
它们对应的\texttt{get<>()}函数都会被调用。
因为定义的\texttt{get<>}函数返回类型是\texttt{auto},所以这3个getter会返回成员的拷贝,
这意味着结构化绑定的变量的地址不同于\texttt{c}中成员的地址。
因此,修改\texttt{c}的值并不会影响绑定变量(反之亦然)。

使用结构化绑定等同于使用\texttt{get<>()}函数的返回值,因此:
\begin{lstlisting}
    std::cout << "f/l/v:    " << f << ' '
              << l << ' ' << v << '\n';
\end{lstlisting}
只是简单的输出变量的值(并不会再次调用getter函数)。另外
\begin{lstlisting}
    std::string s{std::move(f)};
    l = "Waters";
    v += 10;
    std::cout << "f/l/v:    " << f << ' '
              << l << ' ' << v << '\n';
\end{lstlisting}
这段代码修改了绑定变量的值。因此,这段程序总是有如下输出:
\begin{lstlisting}
    f/l/v:    Tim Starr 42
    f/l/v:     Waters 52
    c:        Tim Starr 42
    s:        Tim
\end{lstlisting}
第二行的输出依赖于被move的\texttt{string}的值,一般情况下是空值,但也有可能是其他有效的值。

你也可以在迭代一个元素类型是\texttt{Customer}的\texttt{vector}时使用结构化绑定:
\begin{lstlisting}
    std::vector<Customer> coll;
    ...
    for (const auto& [first, last, val] : coll) {
        std::cout << first << ' ' << last
                  << ": " << val << '\n';
    }
\end{lstlisting}
在这个循环中,因为使用了\texttt{const auto\&}所以不会有\texttt{Customer}被拷贝。
然而,结构化绑定时会调用\texttt{get<>()}函数返回姓和名的拷贝。
之后,循环体内的输出语句中再次使用了结构化绑定,不需要再次调用getter。
最后在每一次迭代结束的时候,拷贝的\texttt{string}会被销毁。

注意对绑定变量使用\texttt{decltype}会推导出变量自身的类型,
不会受到匿名实体的类型修饰符的影响。也就是说这里\texttt{decltype(first)}的类型是
\texttt{const std::string}而不是引用。

\subsubsection*{支持可写结构化绑定}
实现tuple-like API时可以时候用返回\texttt{non-const}引用。这样结构化绑定就变得可写。
设想类\texttt{Customer}提供了读写成员的API\footnote{这个类的设计比较失败,
因为通过成员函数可以直接访问私有成员。然而用来演示怎么支持可写结构化绑定已经足够了}:
\begin{lstlisting}[frame=single, title=lang/customer2.hpp]
    #include <string>
    #include <utility>  // for std::move()

    class Customer {
      private:
        std::string first;
        std::string last;
        long val;
      public:
        Customer (std::string f, std::string l, long v)
            : first{std::move(f))}, last{std::move(l)}, val{v} {
        }
        const std::string& firstname() const {
            return first;
        }
        std::string& firstname() {
            return first;
        }
        const std::string& lastname() const {
            return last;
        }
        long value() const {
            return val;
        }
        long& value() {
            return val;
        }
    };
\end{lstlisting}

为了支持读写,我们需要为常量和非常量引用定义重载的getter:
\begin{lstlisting}[frame=single, title=lang/structbind2.hpp]
    #include "customer2.hpp"
    #include <utility>  // for tuple-like API

    // 为类Customer提供tuple-like API
    template<>
    struct std::tuple_size<Customer> {
        static constexpr int value = 3; // 有3个属性
    };

    template<>
    struct std::tuple_element<2, Customer> {
        using type = long;  // 最后一个属性是long类型
    }
    template<std::size_t Idx>
    struct std::tuple_element<Idx, Customer> {
        using type = std::string;   // 其他的属性是string
    }

    // 定义特化的getter:
    template<std::size_t I> decltype(auto) get(Customer& c) {
        static_assert(I < 3);
        if constexpr (I == 0) {
            return c.firstname();
        }
        else if constexpr (I == 1) {
            return c.lastname();
        }
        else {  // I == 2
            return c.value();
        }
    }
    template<std::size_t I> decltype(auto) get(const Customer& c) {
        static_assert(I < 3);
        if constexpr (I == 0) {
            return c.firstname();
        }
        else if constexpr (I == 1) {
            return c.lastname();
        }
        else {  // I == 2
            return c.value();
        }
    }
    template<std::size_t I> decltype(auto) get(Customer&& c) {
        static_assert(I < 3);
        if constexpr (I == 0) {
            return std::move(c.firstname());
        }
        else if constexpr (I == 1) {
            return std::move(c.lastname());
        }
        else {  // I == 2
            return c.value();
        }
    }
\end{lstlisting}
注意你必须提供这3个版本的特化来分别处理常量对象、非常量对象、可移动对象
\footnote{标准库中还为\texttt{const\&\&}实现了第4个版本的\texttt{get<>()},
这么做是有原因的(见\url{https://wg21.link/lwg2485)}),
但如果只是想支持结构化绑定则不是必须的。}。
为了实现返回引用,你应该使用\texttt{decltype(auto)}
\footnote{\texttt{decltype(auto)}在C++14中引入,
它可以根据表达式的值类别(\emph{value category})来推导(返回)类型。
简单来说,将它设置为返回值类型之后引用会以引用返回,但临时值会以值返回。}。

这里我们又一次使用了\hyperref[ch10]{编译期\texttt{if}语句特性},
这可以让我们的getter的实现变得更加简单。如果没有这个特性,我们必须写出所有的全特化版本,例如:
\begin{lstlisting}
    template<std::size_t> decltype(auto) get(const Customer& c);
    template<std::size_t> decltype(auto) get(Customer& c);
    template<std::size_t> decltype(auto) get(Customer&& c);
    template<> decltype(auto) get<0>(const Customer& c) { return c.firstname(); }
    template<> decltype(auto) get<0>(Customer& c) { return c.firstname(); }
    template<> decltype(auto) get<0>(Customer&& c) { return c.firstname(); }
    template<> decltype(auto) get<1>(const Customer& c) { return c.lastname(); }
    template<> decltype(auto) get<1>(Customer& c) { return c.lastname(); }
    ...
\end{lstlisting}
再次强调,主函数模板声明必须和全特化版本拥有完全相同的签名(包括返回值)。
下面的代码不能通过编译:
\begin{lstlisting}
    template<std::size_t> decltype(auto) get(Customer& c);
    template<> std::string& get<0>(Customer& c) { return c.firstname(); }
    template<> std::string& get<1>(Customer& c) { return c.lastname(); }
    template<> long& get<2>(Customer& c) { return c.value(); }
\end{lstlisting}
你现在可以对\texttt{Customer}类使用结构化绑定了,并且还能通过绑定修改成员的值:
\begin{lstlisting}[frame=single, title=lang/structbind2.cpp]
    #include "structbind2.hpp"
    #include <iostream>

    int main()
    {
        Customer c{"Tim", "Starr", 42};
        auto [f, l, v] = c;
        std::cout << "f/l/v:    " << f << ' '
                  << l << ' ' << v << '\n';

        // 通过引用修改结构化绑定
        auto&& [f2, l2, v2] = c;
        std::string s{std::move(f2)};
        f2 = "Ringo";
        v2 += 10;
        std::cout << "f2/l2/v2: " << f2 << ' '
                  << l2 << ' ' << v2 << '\n';
        std::cout << "c:        " << c.firstname() << ' '
                  << c.lastname() << ' ' << c.value() << '\n';
        std::cout << "s:        " << s << '\n';
    }
\end{lstlisting}
程序的输出如下:
\begin{lstlisting}
    f/l/v:    Tim Starr 42
    f2/l2/v2: Ringo Starr 52
    c:        Ringo Starr 52
    s:        Tim
\end{lstlisting}

\subsection{后记}
结构化绑定最早由Herb Sutter、Bjarne Stroustrup和Gabriel Dos Reis在
\url{https://wg21.link/p0144r0}提出,当时提议使用花括号而不是方括号。
最终被接受的正式提案由Jens Maurer在\url{https://wg21.link/p0217r3}上发表。

    \section{带初始化的\texttt{if}和\texttt{switch}语句}\label{ch2}
\texttt{if}和\texttt{switch}语句现在允许在条件表达式里添加一条初始化语句。
例如,你可以写:
\begin{lstlisting}
    if (status s = check(); s != status::success) {
        return s;
    }
\end{lstlisting}
这里初始化语句:
\begin{lstlisting}
    status s = check();
\end{lstlisting}
初始化了\texttt{s},\texttt{s}将在整个\texttt{if}语句中有效(包括\texttt{else}分支里)。

\subsection{带初始化的\texttt{if}语句}
在\texttt{if}语句的条件表达式里定义的变量将在整个\texttt{if}语句中生效,
包括\emph{then}和\emph{else}部分。例如:
\begin{lstlisting}
    if (std::ofstream strm = getLogStrm(); coll.empty()) {
        strm << "<no data>\n";
    }
    else {
        for (const auto& elem : coll) {
            strm << elem << '\n';
        }
    }
    // strm不再有效
\end{lstlisting}
在整个\texttt{if}语句结束时\texttt{strm}的析构函数会被调用。
另一个例子是关于锁的使用,假设我们要在并发的环境中执行一个依赖某些条件的任务:
\begin{lstlisting}
    if (std::lock_guard<std::mutex> lg{collMutex}; !coll.empty()) {
        std::cout << coll.front() << '\n';
    }
\end{lstlisting}
这个例子中,如果使用\nameref{ch9},可以改写成如下代码:
\begin{lstlisting}
    if (std::lock_guard lg{collMutex}; !coll.empty()) {
        std::cout << coll.front() << '\n';
    }
\end{lstlisting}
上面的代码等价于:
\begin{lstlisting}
    {
        std::lock_guard<std::mutex> lg{collMutex};
        if (!coll.empty()) {
            std::cout << coll.front() << '\n';
        }
    }
\end{lstlisting}
细微的区别在于前者中\texttt{lg}在\texttt{if}语句的作用域之内定义,
和条件语句在相同的作用域。

注意这个特性的效果和传统\texttt{for}循环里的初始化语句完全相同。
上面的例子中为了让\texttt{lock\_guard}生效,必须在初始化语句里明确声明一个变量名,
否则它就是一个临时变量,会在创建之后就立即销毁。因此,初始化一个没有变量名的临时
\texttt{lock\_guard}是一个逻辑错误,因为当执行到条件语句时锁就已经被释放了:
\begin{lstlisting}
    if (std::lock_guard<std::mutex>{collMutex};     // 运行时ERROR
        !coll.empty()) {                    // 锁已经被释放了
        std::cout << coll.front() << '\n';  // 锁已经被释放了
    }
\end{lstlisting}
原则上讲,使用简单的\texttt{\_}作为变量名就已经足够了:
\begin{lstlisting}
    if (std::lock_guard<std::mutex> _{collMutex};   // OK,但是...
        !coll.empty()) {
        std::cout << coll.front() << '\n';
    }
\end{lstlisting}
你也可以同时声明多个变量,并且可以在声明时初始化:
\begin{lstlisting}
    if (auto x = qqq1(), y = qqq2(); x != y) {
        std::cout << "return values " << x << " and " << y << "differ\n";
    }
\end{lstlisting}
或者:
\begin{lstlisting}
    if (auto x{qqq1()}, y{qqq2()}; x != y) {
        std::cout << "return values " << x << " and " << y << "differ\n";
    }
\end{lstlisting}
作为另一个例子,考虑向\texttt{map}或者\texttt{unordered map}插入元素。
你可以像下面这样检查是否成功:
\begin{lstlisting}
    std::map<std::string, int> coll;
    ...
    if (auto [pos, ok] = coll.insert({"new", 42}); !ok) {
        // 如果插入失败,用pos处理错误
        const auto& [key, val] = *pos;
        std::cout << "already there: " << key << '\n';
    }
\end{lstlisting}
这里,我们用了\nameref{ch1}来给返回的\texttt{pos}指向的值声明了新的名称,
而不是使用\texttt{first}和\texttt{second}成员。在C++17之前,相应的处理代码必须像下面这样写:
\begin{lstlisting}
    auto ret = coll.insert({"new", 42});
    if (!ret.second) {
        // 如果插入失败,用ret.first处理错误
        const auto& elem = *(ret.first);
        std::cout << "already there: " << elem.first << '\n';
    }
\end{lstlisting}
注意这个拓展也适用于\nameref{ch10}特性。

\subsection{带初始化的\texttt{switch}语句}
通过使用带初始化的\texttt{switch}语句,我们可以在控制流之前初始化一个对象/实体。
例如,我们可以先声明一个\nameref{ch20.2.3},然后再根据它的类别进行处理:
\begin{lstlisting}
    namespace fs = std::filesystem;
    ...
    switch (fs::path p{name}; status(p).type()) {
        case fs::file_type::not_found:
            std::cout << p << " not found\n";
            break;
        case fs::file_type::directory:
            std::cout << p << ":\n";
            for (const auto& e : std::filesystem::directory_iterator{p}) {
                std::cout << "- " << e.path() << '\n';
            }
            break;
        default:
            std::cout << p << " exists\n";
            break;
    }
\end{lstlisting}
这里,初始化的路径\texttt{p}可以在整个\texttt{switch}语句中使用。

\subsection{后记}
带初始化语句的\texttt{if}和\texttt{switch}语句最早由Thomas Köppe
在\url{https://wg21.link/p0305r0}中提出,一开始只是提到了扩展\texttt{if}语句。
最终被接受的正式提案由Thomas Köpped发表于\url{https://wg21.link/p0305r1}。
    \include{ch3}
    \include{ch4}
    \include{ch5}
    \include{ch6}
    \include{ch7}
    \section{其他语言特性}\label{ch8}
在C++17中还有一些微小的核心语言特性的变更,将在这一章中介绍。

\subsection{嵌套命名空间}
自从2003年第一次提出,到现在C++标准委员会终于同意了以如下方式定义嵌套的命名空间:
\begin{lstlisting}
    namespace A::B::C {
        ...
    }
\end{lstlisting}
等价于:
\begin{lstlisting}
    namespace A {
        namespace B {
            namespace C {
                ...
            }
        }
    }
\end{lstlisting}
注意目前还没有对嵌套内联命名空间的支持。
这是因为\texttt{inline}是作用于最内层还是整个命名空间还有歧义。(两种情况都很有用)

\subsection{有定义的表达式求值顺序}
许多C++书籍里的代码如果按照直觉来看似乎是正确的,但严格上讲它们有可能导致未定义的行为。
一个简单的例子是在字符串中替换一个字串:
\begin{lstlisting}
    std::string s ="I heard it even works if you don't believe";
    s.replace(0, 8, "").replace(s.find("even", 4, "sometimes")
                       .replace(s.find("you don't"), 9, "I");
\end{lstlisting}
通常的假设是前8个字符被空串替换,“even”被“sometimes”替换,“you don't”被“I”替换,
因此结果是:
\begin{lstlisting}[keywordstyle=\color{black}]
    it sometimes works if I believe
\end{lstlisting}
然而在C++17之前最后的结果实际上并没有任何保证。因为查找子串位置的\texttt{find()}
函数可能在需要它们的返回值之前的任意时刻调用,而不是像直觉中的那样从左向右按顺序执行表达式。
事实上,所有的\texttt{find()}调用可能在执行第一次替换之前就全部执行,因此结果变为:

\begin{lstlisting}
    it even worsometimesf youIlieve
\end{lstlisting}
其他的结果也是有可能的:
\begin{lstlisting}
    it sometimes workIdon’t believe
    it even worsometiIdon’t believe
\end{lstlisting}
作为另一个例子,考虑使用输出运算符打印几个相互依赖的值:
\begin{lstlisting}
    std::cout << f() << g() << h();
\end{lstlisting}
通常的假设是依次调用\texttt{f()}、\texttt{g()}、\texttt{h()}函数。
然而这个假设实际上是错误的。\texttt{f()}、\texttt{g()}、\texttt{h()}有可能以任意顺序调用,
当这三个函数的调用顺序会影响返回值的时候可能就会出现奇怪的结果。

作为一个具体的例子,直到C++17之前,下面代码的行为都是未定义的:
\begin{lstlisting}
    i = 0;
    std::cout << ++i << ' ' << --i << '\n';
\end{lstlisting}
在C++17之前,它\emph{\textbf{可能}}会输出\texttt{1 0},但也可能输出\texttt{0 -1}
或者\texttt{0 0},这和\texttt{i}是\texttt{int}还是用户自定义类型无关(不过对于基本类型,
编译器一般会在这种情况下给出警告)。

为了解决这种未定义的问题,C++17标准重新定义了\emph{一些}运算符的的求值顺序,
因此这些运算符现在有了固定的求值顺序:
\begin{itemize}[leftmargin=*]
    \item 对于运算
    \begin{lstlisting}
    e1 [ e2 ]
    e1 . e2
    e1 .* e2
    e1 ->* e2
    e1 << e2
    e1 >> e2
    \end{lstlisting}
    \emph{e1}现在保证一定会在\emph{e2}之前求值,因此求值顺序是从左向右。
    然而,注意同一个函数调用中的不同参数的计算顺序仍然是未定义的。也就是说:
    \begin{lstlisting}
    e1.f(a1, a2, a3);
    \end{lstlisting}
    中的\texttt{e1.f}保证会在\texttt{a1}、\texttt{a2}、\texttt{a3}之前求值。
    但\texttt{a1}、\texttt{a2}、\texttt{a3}的求值顺序仍是未定义的。
    \item 所有的赋值运算
    \begin{lstlisting}
    e2 = e1
    e2 += e1
    e2 *= e1
    ...
    \end{lstlisting}
    中右侧的\texttt{e1}现在保证一定会在左侧的\texttt{e2}之前求值。
    \item 最后,类似于如下的\texttt{new}表达式
    \begin{lstlisting}
        new Type(e)
    \end{lstlisting}
    中保证内存分配的操作在对\texttt{e}求值之前发生。
    新的对象的初始化操作保证在第一次使用该对象之前完成。
\end{itemize}
所有这些保证适用于所有基本类型和自定义类型。

因此,自从C++17起
\begin{lstlisting}
    std::string s ="I heard it even works if you don't believe";
    s.replace(0, 8, "").replace(s.find("even"), 4, "always")
        .replace(s.find("don't believe"), 13, "use C++17");
\end{lstlisting}
保证将会把\texttt{s}的值修改为:
\begin{lstlisting}[keywordstyle=\color{black}]
    it always works if you use C++17
\end{lstlisting}
因为现在每个\texttt{find()}之前的替换操作现在都保证会在\texttt{find()}调用之前完成。

另一个例子,如下语句:
\begin{lstlisting}
    i = 0;
    std::cout << ++i << ' ' << --i << '\n';
\end{lstlisting}
对于任意类型的\texttt{i}都保证输出是\texttt{1 0}。

然而,其他大多数运算符的运算顺序仍然是未知的。例如:
\begin{lstlisting}
    i = i++ + i;    // 仍然是未定义的行为
\end{lstlisting}
这里,最右侧的\texttt{i}可能在\texttt{i}自增之前求值也可能在自增之后求值。

新的表达式求值顺序的另一个应用是在参数之前\hyperref[ch11.2.1]{插入空格的函数}。

\paragraph{向后的不兼容性}
新的有定义的求值顺序可能会影响现有程序的输出。例如,考虑如下程序:
\begin{lstlisting}[frame=single, title=lang/evalexcept.cpp]
    #include <iostream>
    #include <vector>

    void print10elems(const std::vector<int>& v) {
        for (int i = 0; i < 10; ++i) {
            std::cout << "value: " << v.at(i) << '\n';
        }
    }

    int main()
    {
        try {
            std::vector<int> vec{7, 14, 21, 28};
            print10elems(vec);
        }
        catch (const std::exception& e) {   // 处理标准异常
            std::cerr << "EXCEPTION: " << e.what() << '\n';
        }
        catch (...) {   // 处理任何其他异常
            std::cerr << "EXCEPTION of unknown type\n";
        }
    }
\end{lstlisting}
因为这个程序中的\texttt{vector<>}只有4个元素,因此在\texttt{print10elems()}的循环中
使用无效的索引调用\texttt{at()}时将会抛出异常:
在C++17之前,输出可能是:
\begin{lstlisting}
    value: 7
    value: 14
    value: 21
    value: 28
    EXCEPTION: ...
\end{lstlisting}
因为\texttt{at()}允许在输出\texttt{value:}之前调用,
所以当索引错误时可以跳过开头的\texttt{value:}输出。
\footnote{较旧版本的GCC或者Visual C++的行为就是这样的。}

自从C++17以后,输出保证是:
\begin{lstlisting}
    value: 7
    value: 14
    value: 21
    value: 28
    value: EXCEPTION: ...
\end{lstlisting}
因为现在\texttt{value:}的输出保证在\texttt{at()}调用之前。

\subsection{更宽松的用整型初始化枚举值的规则}
对于一个有固定基础类型的枚举类型变量,自从C++17开始可以用一个整型值直接进行列表初始化。
这可以用于带有明确类型的无作用域枚举和所有有作用域的枚举,因为它们都有默认的基础类型:
\begin{lstlisting}
    // 带有明确基础类型的无作用域枚举类型
    enum MyInt : char { };
    MyInt i1{42};       // 自从C++17起OK(C++17以前ERROR)
    MyInt i2 = 42;      // 仍然ERROR
    MyInt i3(42);       // 仍然ERROR
    MyInt i4 = {42};    // 仍然ERROR

    // 带有默认基础类型的有作用域枚举
    enmu class Weekday { mon, tue, wed, thu, fri, sat, sun };
    Weekday s1{0};      // 自从C++17起OK(C++17以前ERROR)
    Weekday s2 = 0;     // 仍然ERROR
    Weekday s3(0);      // 仍然ERROR
    Weekday s4 = {0};   // 仍然ERROR
\end{lstlisting}
如果\texttt{Weekday}有明确的基础类型的话结果完全相同:
\begin{lstlisting}
    // 带有明确基础类型的有作用域枚举
    enum class Weekday : char { mon, tue, wed, thu, fri, sat, sun };
    Weekday s1{0};      // 自从C++17起OK(C++17以前ERROR)
    Weekday s2 = 0;     // 仍然ERROR
    Weekday s3(0);      // 仍然ERROR
    Weekday s4 = {0};   // 仍然ERROR
\end{lstlisting}
对于\emph{没有}明确基础类型的无作用域枚举类型(没有\texttt{class}的\texttt{enum}),
你仍然不能使用列表初始化:
\begin{lstlisting}
    enum Flag { bit1=1, bit2=2, bit3=4 };
    Flag f1{0};     // 仍然ERROR
\end{lstlisting}
注意列表初始化不允许窄化,所以你不能传递一个浮点数:
\begin{lstlisting}
    enum MyInt : char { };
    MyInt i5{42.2}; // 仍然ERROR
\end{lstlisting}
一个定义新的整数类型的技巧是简单的定义一个以某个已有整数类型作为基础类型的枚举类型,
就像上面例子中的\texttt{MyInt}一样。
这个特性的动机之一就是为了支持这个技巧,如果没有这个特性,在不进行转换的情况下将无法初始化新的对象。

事实上自从C++17起标准库提供的\nameref{ch18}就直接使用了这个特性。

\subsection{修正了\texttt{auto}类型的列表初始化}
自从在C++11中引入了花括号\emph{统一}初始化之后,
每当使用\texttt{auto}代替明确类型进行列表初始化时就会出现一些意料之外的不一致的结果:
\begin{lstlisting}
    int x{42};       // 初始化一个int
    int y{1, 2, 3};  // ERROR
    auto a{42};      // 初始化一个std::initializer_list<int>
    auto b{1, 2, 3}; // OK:初始化一个std::initializer_list<int>
\end{lstlisting}
这些\emph{直接}使用列表初始化(没有使用=)时的不一致行为现在已经被修复了。
因此如下代码的行为变成了:
\begin{lstlisting}
    int x{42};       // 初始化一个int
    int y{1, 2, 3};  // ERROR
    auto a{42};      // 现在初始化一个int
    auto b{1, 2, 3}; // 现在ERROR
\end{lstlisting}
注意这是一个\textbf{破坏性的更改(breaking change)},因为它可能导致很多代码的行为在
无声无息中发生改变。因此,支持了这个变更的编译器现在即使在C++11模式下也会启用这个变更。
对于主流编译器,接受这个变更的版本分别是Visual Studio 2015,g++5,clang3.8。

注意当使用\texttt{auto}进行\emph{拷贝}列表初始化(使用了=)时仍然是初始化一个
\texttt{std::initializer\_list<>}:
\begin{lstlisting}
    auto c = {42};  // 仍然初始化一个std::initializer_list<int>
    auto d = {1, 2, 3};   // 仍然OK:初始化一个std::initializer_list<int>
\end{lstlisting}
因此,现在直接初始化(没有=)和拷贝初始化(有=)之间又有了显著的不同:
\begin{lstlisting}
    auto a{42};     // 现在初始化一个int
    auto c = {42};  // 仍然初始化一个std::initializer_list<int>
\end{lstlisting}
这也是更推荐使用直接列表初始化(没有=的花括号初始化)的原因之一。

\subsection{十六进制浮点数字面量}
C++17允许指定十六进制浮点数字面量(有些编译器甚至在C++17之前就已经支持)。
当需要一个精确的浮点数表示时这个特性非常有用(如果直接用十进制的浮点数字面量不保证
存储的实际精确值是多少)。

例如:
\begin{lstlisting}
    #include <iostream>
    #include <iomanip>

    int main()
    {
        // 初始化浮点数
        std::initializer_list<double> values {
            0x1p4,          // 16
            0xA,            // 10
            0xAp2,          // 40
            5e0,            // 5
            0x1.4p+2,       // 5
            1e5,            // 100000
            0x1.86Ap+16,    // 100000
            0xC.68p+2,      // 49.625
        };

        // 分别以十进制和十六进制打印出值:
        for (double d : values) {
            std::cout << "dec: " << std::setw(6)
                      << std::defaultfloat << d << "  hex: "
                      << std::hexfloat << d << '\n';
        }
    }
\end{lstlisting}
程序通过使用已有的和新增的十六进制浮点记号定义了不同的浮点数值。
新的记号是一个以2为基数的科学记数法记号:
\begin{itemize}[leftmargin=*]
    \item 有效数字/尾数用十六进制书写
    \item 指数部分用十进制书写,表示乘以2的n次幂
\end{itemize}
例如\texttt{0xAp2}的值为40($10\times2^2$)。这个值也可以被写作\texttt{0x1.4p+5},
也就是$1.25\times32$(0.4是十六进制的分数,等于十进制的0.25,$2^5=32$)。

程序的输出如下:
\begin{lstlisting}
    dec:     16  hex: 0x1p+4
    dec:     10  hex: 0x1.4p+3
    dec:     40  hex: 0x1.4p+5
    dec:      5  hex: 0x1.4p+2
    dec:      5  hex: 0x1.4p+2
    dec: 100000  hex: 0x1.86ap+16
    dec: 100000  hex: 0x1.86ap+16
    dec: 49.625  hex: 0x1.8dp+5
\end{lstlisting}
就像上例展示的一样,十六进制浮点数的记号很早就存在了,
因为输出流使用的\texttt{std::hexfloat}操作符自从C++11起就已经存在了。

\subsection{UTF-8字符字面量}
自从C++11起,C++就已经支持以\texttt{u8}为前缀的UTF-8字符串字面量。
然而,这个前缀不能用于字符字面量。C++17修复了这个问题,所以现在可以这么写:
\begin{lstlisting}
    auto c = u8'6'; // UTF-8编码的字符6
\end{lstlisting}
在C++17中,\texttt{u8'6'}的类型是\texttt{char},在C++20中可能会变为\texttt{char8\_t},
因此这里使用\texttt{auto}会更好一些。

通过使用该前缀现在可以保证字符值是UTF-8编码。你可以使用所有的7位的US-ASCII字符,
这些字符的UTF-8表示和US-ASCII表示完全相同。
也就是说,\texttt{u8'6'}也是有效的以7位US-ASCII表示的字符'6'
(也是有效的ISO Latin-1、ISO-8859-15、基本Windows字符集中的字符)。
\footnote{ISO Latin-1的正式命名为ISO-8859-1,而为了包含欧元符号€引入的字符集
ISO-8859-15也被命名为ISO Latin-9。}
通常情况下你的源码字符被解释为US-ASCII或者UTF-8的结果是一样的,所以这个前缀并不是必须的。
\texttt{c}的值永远是\texttt{54}(十六进制\texttt{36})。

这里给出一些背景知识来说明这个前缀的必要性:对于源码中的字符和字符串字面量,
C++标准化了你可以使用的字符而不是这些字符的值。这些值取决于\emph{源码字符值}。
当编译器为源码生成可执行程序时它使用\emph{运行字符集}。源码字符集几乎总是7位的
US-ASCII编码,而运行字符集通常是相同的。这意味着在任何C++程序中,所有相同的字符和字符串字面量
(不管有没有\texttt{u8}前缀)总是有相同的值。

然而,在一些特别罕见的场景中并不是这样的。例如,在使用EBCDIC字符集的旧的IBM机器上,字符'6'
的值将是246(十六进制为F6)。在一个使用EBCDIC字符集的程序中上面的字符\texttt{c}的值将是
246而不是54,如果在UTF-8编码的平台上运行这个程序可能会打印出字符ö,这个字符在ISO/IEC 8859-x
编码中的值为246.在这种情况下,这个前缀就是必须的。

注意\texttt{u8}只能用于单个字符,并且该字符的UTF-8编码必须只占一个字节。一个如下的初始化:
\begin{lstlisting}
    char c = u8'ö';
\end{lstlisting}
是不允许的,因为德语的曲音字符ö的UTF-8编码是两个字节的序列,分别是195和182(十六进制为C3 B6)。

因此,字符和字符串字面量现在接受如下前缀:
\begin{itemize}[leftmargin=*]
    \item u8用于单字节US-ASCII和UTF-8编码
    \item u用于两字节的UTF-16编码
    \item U用于四字节的UTF-32编码
    \item L用于没有指定编码,可能是两个或者四个字节的宽字符集
\end{itemize}

\subsection{异常声明作为类型的一部分}






































    \part{模板特性}\label{part2}
    这一部分介绍了C++17为泛型编程(即template)提供的新的语言特性。

    我们首先从类模板参数推导开始,这一特性只影响模板的使用。之后的章节会介绍为编写泛型代码(函数模板,
    类模板,泛型库等)的程序员提供的新特性。

    \section{类模板参数推导}\label{ch9}

\subsection{使用类模板参数推导}

\subsubsection{默认以拷贝方式推导}

\subsubsection{推导lambda的类型}

\subsubsection{没有类模板部分参数推导}

\subsubsection{使用类模板参数推导代替快捷函数}

\subsection{推导指引}

\subsubsection{使用推导指引强制类型退化}

\subsubsection{非模板推导指引}

\subsubsection{推导指引与构造函数冲突}

\subsubsection{\texttt{explicit}推导指引}

\subsubsection{聚合体的推导指引}

\subsubsection{标准推导指引}

\paragraph{pair和tuple的推导指引}

\paragraph{从迭代器推导}

\paragraph{\texttt{std::array<>}推导}\label{ch9.2.6.3}

\paragraph{(Unordered) Map推导}

\paragraph{智能指针没有推导指引}

\subsection{后记}
    \section{编译期\texttt{if}语句}\label{ch10}
    \section{折叠表达式}\label{ch11}
自从C++17起,有一个新的特性可以计算对参数包中的\emph{所有}参数应用一个二元运算符的结果。

例如,下面的函数将会返回所有参数的总和:
\begin{lstlisting}
    template<typename... T>
    auto foldSum (T... args) {
        return (... + args);    //((arg1 + arg2) + arg3)...
    }
\end{lstlisting}
注意返回语句中的括号是折叠表达式的一部分,不能被省略。

如下调用:
\begin{lstlisting}
    foldSum(47, 11, val, -1);
\end{lstlisting}
会把模板实例化为:
\begin{lstlisting}
    return 47 + 11 + val + -1;
\end{lstlisting}
如下调用:
\begin{lstlisting}
    foldsum(std::string("hello"), "world", "!");
\end{lstlisting}
会把模板实例化为:
\begin{lstlisting}
    return std::string("hello") + "world" + "!";
\end{lstlisting}
注意折叠表达式里参数的位置很重要(可能看起来还有些反直觉)。如下写法:
\begin{lstlisting}
    (... + args)
\end{lstlisting}
会展开为:
\begin{lstlisting}
    ((arg1 + arg2) + arg3) ...
\end{lstlisting}
这意味着折叠表达式会以后递增式重复展开。你也可以写:
\begin{lstlisting}
    (args + ...)
\end{lstlisting}
这样就会前递增式展开,因此结果会变为:
\begin{lstlisting}
    (arg1 + (arg2 + arg3)) ...
\end{lstlisting}

\subsection{折叠表达式的动机}
折叠表达式的出现让我们不必再用递归实例化模板的方式来处理参数包。
在C++17之前,你必须实现为:
\begin{lstlisting}
    template<typename T>
    auto foldSumRec (T arg) {
        return arg;
    }
    template<typename T1, typename... Ts>
    auto foldSumRec (T1 arg1, Ts... otherArgs) {
        return arg1 + foldSumRec(otherArgs...);
    }
\end{lstlisting}
这样的实现不仅写起来麻烦,对C++编译器来说也很难处理。使用如下写法:
\begin{lstlisting}
    template<typename... T>
    auto foldSum (T... args) {
        return (... + args);    // arg1 + arg2 + arg3
    }
\end{lstlisting}
能显著的减少程序员和编译器的工作量。

\subsection{使用折叠表达式}\label{ch11.2}
给定一个参数\emph{args}和一个操作符\emph{op},C++17允许我们这么写:
\begin{itemize}[leftmargin=*]
    \item \emph{\textbf{一元左折叠}}\\
    ( \ldots \, \emph{\textbf{op} args} )\\
    将会展开为:\emph{((arg1 \textbf{op} arg2) \textbf{op} arg3) \textbf{op} \ldots}
    \item \emph{\textbf{一元右折叠}}\\
    ( \emph{args \textbf{op}} \ldots \,)\\
    将会展开为:\emph{arg1 \textbf{op} (arg2 \textbf{op} \ldots \, (argN-1 \textbf{op} argN))}
\end{itemize}
括号是必须的,然而,括号和省略号(\ldots)之间并不需要用空格分隔。

左折叠和右折叠的不同比想象中更大。例如,当你使用\textbf{+}时可能会产生不同的效果。
使用左折叠时:
\begin{lstlisting}
    template<typename... T>
    auto foldSumL(T... args) {
        return (... + args);    // ((arg1 + arg2) + arg3)...
    }
\end{lstlisting}
如下调用
\begin{lstlisting}
    foldSumL(1, 2, 3);
\end{lstlisting}
会求值为
\begin{lstlisting}
    ((1 + 2) + 3)
\end{lstlisting}
这意味着下面的例子能够通过编译:
\begin{lstlisting}
    std::cout << foldSumL(std::string("hello"), "world", "!")
              << '\n';  // OK
\end{lstlisting}
记住对字符串而言只有两侧至少有一个是\texttt{std::string}时才能使用\texttt{+}。
使用作折叠式,会首先计算
\begin{lstlisting}
    std::string("hello") + "world"
\end{lstlisting}
这将返回一个\texttt{std::string},因此再加上字符串字面量\texttt{"!"}是有效的。

然而,如下调用
\begin{lstlisting}
    std::cout << foldSumL("hello", "world", std::string("!"))
              << '\n';  // ERROR
\end{lstlisting}
将不能通过编译,因为它会求值为
\begin{lstlisting}
    ("hello" + "world") + std::string("!");
\end{lstlisting}
然而把两个字符串字面量相加是错误的。

然而如果我们把实现修改为:
\begin{lstlisting}
    template<typename... T>
    auto foldSumR(T... args) {
        return (args + ...);    // (arg1 + (arg2 + arg3))...
\end{lstlisting}
那么如下调用
\begin{lstlisting}
    foldSumR(1, 2, 3)
\end{lstlisting}
将求值为
\begin{lstlisting}
    (1 + (2 + 3))
\end{lstlisting}
这意味着下面的例子不能再通过编译:
\begin{lstlisting}
    std::cout << foldSumR(std::string("hello"), "world", "!")
              << '\n';  // ERROR
\end{lstlisting}
然而如下调用现在反而可以编译了:
\begin{lstlisting}
    std::cout << foldSumR("hello", "world", std::string("!"))
              << '\n';  // OK
\end{lstlisting}
在任何情况下,从左向右求值都是符合直觉的。
因此,更推荐使用左折叠的语法:
\begin{lstlisting}
    (... + args);       // 推荐的折叠表达式语法
\end{lstlisting}

\subsubsection{处理空参数包}\label{ch11.2.1}
当使用折叠表达式处理空参数包时,将遵循如下规则:
\begin{itemize}[leftmargin=*]
    \item 如果使用了\texttt{\&\&}运算符,值为\texttt{true}。
    \item 如果使用了\texttt{||}运算符,值为\texttt{false}。
    \item 如果使用了逗号运算符,值为\texttt{void()}。
    \item 使用所有其他的运算符,都会引发格式错误
\end{itemize}
对于所有其他的情况,你可以添加一个初始值:
给定一个参数包\emph{args},一个初始值\emph{value},一个操作符\emph{op},
C++17允许我们这么写:
\begin{itemize}[leftmargin=*]
    \item \emph{\textbf{二元左折叠}}\\
    ( \emph{value \textbf{op} \ldots \, \textbf{op} args} )\\
    将会展开为:\emph{(((value \textbf{op} arg1) \textbf{op} arg2) \textbf{op} arg3) \textbf{op} \ldots}
    \item \emph{\textbf{二元右折叠}}\\
    ( \emph{args \textbf{op} \ldots \, \textbf{op} value} )\\
    将会展开为:\emph{arg1 \textbf{op} (arg2 \textbf{op} \ldots \, (argN \textbf{op} value))}
\end{itemize}
省略号两侧的\emph{op}必须相同。

例如,下面的定义在进行加法时允许传递一个空参数包:
\begin{lstlisting}
    template<typename... T>
    auto foldSum (T... s) {
        return (0 + ... + s);   // 即使sizeof...(s)==0也能工作
    }
\end{lstlisting}
从概念上讲,不管\texttt{0}是第一个还是最后一个操作数应该和结果无关:
\begin{lstlisting}
    template<typename... T>
    auto foldSum (T... s) {
        return (s + ... + 0);   // 即使sizeof...(s)==0也能工作
    }
\end{lstlisting}
然而,对于一元折叠表达式来说,\hyperref[ch11.2]{不同的求值顺序比想象中的更重要}。
对于二元表达式来说,也更推荐左折叠的方式:
\begin{lstlisting}
    (val + ... + args);         // 推荐的二元折叠表达式语法
\end{lstlisting}
有时候第一个操作数是特殊的,比如下面的例子:
\label{输出任意个参数的print}
\begin{lstlisting}
    template<typename... T>
    void print(const T&... args)
    {
        (std::cout << ... << args) << '\n';
    }
\end{lstlisting}
这里,传递给\texttt{print()}的第一个参数输出之后将返回输出流,所以后面的参数可以继续输出。
其他的实现可能不能编译或者产生一些意料之外的结果。例如,
\begin{lstlisting}
    std::cout << (args << ... << '\n');
\end{lstlisting}
类似\texttt{print(1)}的调用可以编译,但会打印出1左移\texttt{'\textbackslash n'}位之后的值,
\texttt{'\textbackslash n'}的值通常是10,所以结果是1024。

注意在这个\texttt{print()}的例子中,两个参数之间没有输出空格字符。
因此,如下调用\texttt{print("hello", 42, "world")}将会打印出:
\begin{lstlisting}
    hello42world
\end{lstlisting}
为了用空格分隔传入的参数,
你需要一个帮助函数来确保除了第一个参数之外的剩余参数输出前都先输出一个空格。
例如,使用如下的模板\texttt{spaceBefore()}可以做到这一点:
\inputcodefile{tmpl/addspace.hpp}
这里,折叠表达式
\begin{lstlisting}
    (std::cout << ... << spaceBefore(args))
\end{lstlisting}
将会展开为:
\begin{lstlisting}
    std::cout << spaceBefore(arg1) << spaceBefore(arg2) << ...
\end{lstlisting}
因此,对于参数包中的每一个参数\texttt{args},都会调用帮助函数,
在输出参数之前先输出一个空格到\texttt{std::cout}。
为了确保不会对第一个参数调用帮助函数,我们添加了额外的模板参数对第一个参数进行单独处理。

注意要想让参数包正确输出需要确保对每个参数调用\texttt{spaceBefore()}之前左侧的所有输出都已经完成。
得益于操作符\texttt{<<}的\nameref{ch8.2},自从C++17起将保证行为正确:

我们也可以使用lambda来在\texttt{print()}内定义\texttt{spaceBefore()}:
\begin{lstlisting}
    template<typename First, typename... Args>
    void print (const First& firstarg, const Args&... args) {
        std::cout << firstarg;
        auto spaceBefore = [](const auto& arg) {
            std::cout << ' ';
            return arg;
        };
        (std::cout << ... << spaceBefore(args)) < '\n';
    }
\end{lstlisting}
然而,注意默认情况下lambda以值返回对象,这意味着会创建参数的不必要的拷贝。
解决方法是显式指明返回类型为\texttt{const auto\&}或者\texttt{decltype(auto)}:
\begin{lstlisting}
    template<typename First, typename... Args>
    void print (const First& firstarg, const Args&... args) {
        std::cout << firstarg;
        auto spaceBefore = [](const auto& arg) -> const auto& {
            std::cout << ' ';
            return arg;
        };
        (std::cout << ... << spaceBefore(args)) << '\n';
    }
\end{lstlisting}
如果你不能把他们写在一个表达式里那么C++就不是C++了:
\begin{lstlisting}
    template<typename First, typename... Args>
    void print (const First& firstarg, const Args&... args) {
        std::cout << firstarg;
        (std::cout << ... << [](const auto& arg) -> decltype(auto) {
                                std::cout << ' ';
                                return arg;
                             }(args)) << '\n';
    }
\end{lstlisting}
不过,一个更简单的实现\texttt{print()}的方法是使用一个lambda输出空格和参数,
然后在一元折叠表达式里使用它:
\footnote{感谢Barry Revzin指出这一点。}
\begin{lstlisting}
    template<typename First, typename... Args>
    void print(First first, const Args&... args) {
        std::cout << first;
        auto outWithSpace = [](const auto& arg) {
                                std::cout << ' ' << arg;
                            };
        (... , outWithSpace(args));
        std::cout << '\n';
    }
\end{lstlisting}
通过使用新的\hyperref[ch13.1.1]{\texttt{auto}模板参数},我们可以使\texttt{print()}
变得更加灵活:可以把间隔符定义为一个参数,这个参数可以是一个字符、一个字符串或者其它任何可打印的类型。

\subsubsection{支持的运算符}
你可以对除了\texttt{.}、\texttt{->}、\texttt{[]}之外的所有二元运算符使用折叠表达式。

\subsubsection*{折叠函数调用}
折叠表达式可以用于逗号运算符,这样就可以在一条语句里进行多次函数调用。
也就是说,你现在可以简单写出如下实现:
\begin{lstlisting}
    template<typename... Types>
    void callFoo(const Types&... args)
    {
        ...
        (... , foo(args));  // 调用foo(arg1),foo(arg2),foo(arg3),...
    }
\end{lstlisting}
来对所有参数调用函数\texttt{foo()}。

另外,如果需要支持移动语义:
\begin{lstlisting}
    template<typename... Types>
    void callFoo(Types&&... args)
    {
        ...
        (... , foo(std::forward<Types>(args))); // 调用foo(arg1),foo(arg2),...
    }
\end{lstlisting}

如果\texttt{foo()}函数返回的类型重载了逗号运算符,那么代码行为可能会改变。
为了保证这种情况下代码依然安全,你需要把返回值转换为\texttt{void}:
\begin{lstlisting}
    template<typename... Types>
    void callFoo(const Types&... args)
    {
        ...
        (... , (void)foo(std::forward<Types>(args))); // 调用foo(arg1),foo(arg2),...
    }
\end{lstlisting}
注意自然情况下,对于逗号运算符不管我们是左折叠还是右折叠都是一样的。
函数调用们总是会从左向右执行。如下写法:
\begin{lstlisting}
    (foo(args) , ...);
\end{lstlisting}
中的括号只是把后边的调用括在一起,因此首先是第一个\texttt{foo()}调用,
之后是被括起来的两个\texttt{foo()}调用:
\begin{lstlisting}
    foo(arg1) , (foo(arg2) , foo(arg3));
\end{lstlisting}
然而,因为逗号表达式的求值顺序通常是自左向右,所以第一个调用通常发生在括号里的的后两个调用之前,
并且括号里左侧的调用在右侧的调用之前。
\footnote{如果重载逗号运算符,你可以改变它的求值顺序,这可能影响左折叠和右折叠的求值顺序。}

不过,因为左折叠更符合自然的求值顺序,因此在使用折叠表达式进行多次函数调用时还是推荐使用左折叠。

\subsubsection*{组合哈希函数}
另一个使用逗号折叠表达式的例子是组合哈希函数。可以用如下的方法完成:
\begin{lstlisting}
    template<typename T>
    void hashCombine (std::size_t& seed, const T& val)
    {
        seed ^= std::hash<T>()(val) + 0x9e3779b9 + (seed<<6) + (seed>>2);
    }

    template<typename... Types>
    std::size_t combinedHashValue (const Types&... args)
    {
        std::size_t seed = 0;               // 初始化seed
        (... , hashCombine(seed, args));    // 链式调用hashCombine()
        return seed;
    }
\end{lstlisting}
如下调用
\begin{lstlisting}
    combinedHashValue ("Hi", "World", 42);
\end{lstlisting}
函数中的折叠表达式将被展开为:
\begin{lstlisting}
    hashCombine(seed, "Hi"), (hashCombine(seed, "World"), hashCombine(seed, 42));
\end{lstlisting}
有了这些定义,我们现在可以轻易的定义出一个新的哈希函数,
并将这个函数用于某一个类型例如\texttt{Customer}的无序set或无序map:
\begin{lstlisting}
    sturct CustomerHash
    {
        std::size_t operator() (const Customer& c) const {
            return combinedHashValue(c.getFirstname(), c.getLastname(),
                                     c.getValue());
        }
    };

    std::unordered_set<Customer, CustomerHash> coll;
    std::unordered_map<Customer, std::string, CustomerHash> map;
\end{lstlisting}

\subsubsection*{折叠基类的函数调用}
折叠表达式可以在更复杂的场景中使用。例如,你可以折叠逗号表达式来调用可变数量基类的成员函数:
\inputcodefile{tmpl/foldcalls.cpp}
这里
\begin{lstlisting}
    template<typename... Bases>
    class MultiBase : private Bases...
    {
        ...
    };
\end{lstlisting}
允许我们用可变数量的基类初始化对象:
\begin{lstlisting}
    MultiBase<A, B, C> mb;
\end{lstlisting}
进一步通过
\begin{lstlisting}
    (... , Bases::print());
\end{lstlisting}
这个折叠表达式将展开为调用每一个基类中的\texttt{print}。
也就是说,这条语句会被展开为如下代码:
\begin{lstlisting}
    (A::print(), B::print()), C::print();
\end{lstlisting}

\subsubsection*{折叠路径遍历}
你也可以使用折叠表达式通过运算符\texttt{->*}遍历一个二叉树中的路径。
考虑下面的递归数据结构:
\inputcodefile{tmpl/foldtraverse.hpp}
这里,
\begin{lstlisting}
    (np ->* ... ->* paths)
\end{lstlisting}
使用了折叠表达式以\texttt{np}为起点遍历可变长度的路径,可以像下面这样使用这个函数:
\inputcodefile{tmpl/foldtraverse.cpp}
当调用
\begin{lstlisting}
    Node::traverse(root, Node::left, Node::right);
\end{lstlisting}
时折叠表达式将展开为:
\begin{lstlisting}
    root ->* Node::left ->* Node::right
\end{lstlisting}
结果等价于
\begin{lstlisting}
    root -> subLeft -> subRight
\end{lstlisting}

\subsubsection{使用折叠表达式处理类型}
通过使用类型特征,我们也可以使用折叠表达式来处理模板参数包(任意数量的模板类型参数)。
例如,你可以使用折叠表达式来判断一些类型是否相同:
\inputcodefile{tmpl/ishomogeneous.hpp}
类型特征\texttt{IsHomogeneous<>}可以像下面这样使用:
\begin{lstlisting}
    IsHomogeneous<int, MyType, decltype(42)>::value
\end{lstlisting}
这种情况下,折叠表达式将会展开为:
\begin{lstlisting}
    std::is_same_v<int, MyType> && std::is_same_v<int, decltype(42)>
\end{lstlisting}
函数模板\texttt{isHomogeneous<>()}可以像下面这样使用:
\begin{lstlisting}
    isHomogeneous(43, -1, "hello", nullptr)
\end{lstlisting}
在这种情况下,折叠表达式将会展开为:
\begin{lstlisting}
    std::is_same_v<int, int> && std::is_same_v<int, const char*>
        && is_same_v<int, std::nullptr_t>
\end{lstlisting}
像通常一样,运算符\texttt{\&\&}会短路求值(出现第一个\texttt{false}时就会停止运算)。

标准库里\hyperref[ch9.2.6.3]{\texttt{std::array<>}的推导指引}就使用了这个特性。

\subsection{后记}
折叠表达式特性最早由Andrew Sutton和Richard Smith在\url{https://wg21.link/n4191}中提出。
最终被接受的提案由Andrew Sutton和Richard Smith发表于\url{https://wg21.link/n4295}。
之后对运算符\texttt{*、+、\&、|}处理空参数包的支持由Thibaut Le Jehan在
\url{https://wg21.link/p0036}中移除。

\setcounter{footnote}{0}

    \chapter{扩展的using声明}\label{ch14}
using声明扩展之后可以支持逗号分隔的名称,也可以支持参数包。

例如,你现在可以这么写:
\begin{lstlisting}
    class Base {
      public:
        void a();
        void b();
        void c();
    };

    class Derived : private Base {
      public:
        using Base::a, Base::b, Base::c;
    };
\end{lstlisting}
在C++17之前,你需要使用3个using声明分别进行声明。

\section{使用变长的using声明}\label{ch14.1}
逗号分隔的using声明允许你用泛型代码从可变数量的所有基类中派生同一种运算。

这项技术的一个很酷的应用是创建一个重载的lambda的集合。通过如下定义:
\inputcodefile{tmpl/overload.hpp}
你可以像下面这样重载两个lambda:\label{重载的两倍lambda}
\begin{lstlisting}
    auto twice = overload {
                    [](std::string& s) { s += s; },
                    [](auto& v) { v *= 2; }
                 };
\end{lstlisting}
这里,我们创建了一个\texttt{overload}类型的对象,并且提供了\hyperref[ch9]{推导指引}
来根据lambda的类型推导出\texttt{overload}的基类的类型。
并且我们使用了\hyperref[ch4]{聚合体初始化}
来调用每个lambda生成的闭包类型的拷贝构造函数来初始化基类子对象。

上例中的using声明使得\texttt{overload}类型可以同时访问所有子类中的函数调用运算符。
如果没有这个using声明,两个基类会产生同一个成员函数\texttt{operator()}的重载,
这将会导致歧义。
\footnote{clang和Visual C++都不会把不同基类中不同类型的同名函数当作歧义处理,
所以这个例子中其实不需要\texttt{using}。然而,这段代码如果没有using声明的将不具备可移植性。}

最后,如果你传递一个字符串参数将会调用第一个重载,其他类型(操作符\texttt{*=}有效的类型)
将会调用第二个重载:
\begin{lstlisting}
    int i = 42;
    twice(i);
    std::cout << "i: " << i << '\n';    // 打印出:84
    std::string s = "hi";
    twice(s);
    std::cout << "s: " << s << '\n';    // 打印出:hihi
\end{lstlisting}
这项技术的另一个应用是\hyperref[ch16.3.3.4]{\texttt{std::variant}访问器}。

\section{使用变长using声明继承构造函数}
除了逐个声明继承构造函数之外,现在还支持如下的方式:
你可以声明一个可变参数类模板\texttt{Multi},让它继承每一个参数类型的基类:
\inputcodefile{tmpl/using2.hpp}
有了所有基类构造函数的using声明,你可以继承每个类型对应的构造函数。

现在,当使用不同类型声明\texttt{Multi<>}时:
\begin{lstlisting}
    using MultiISB = Multi<int, std::string, bool>;
\end{lstlisting}
你可以使用每一个相应的构造函数来声明对象:
\begin{lstlisting}
    MultiISB m1 = 42;
    MultiISB m2 = std::string("hello");
    MultiISB m3 = true;
\end{lstlisting}
根据新的语言规则,每一个初始化会调用匹配基类的相应构造函数和所有其他基类的默认构造函数。因此:
\begin{lstlisting}
    MultiISB m2 = std::string("hello");
\end{lstlisting}
会调用\texttt{Base<int>}的默认构造函数,\texttt{Base<std::string>}的字符串构造函数,
\texttt{Base<bool>}的默认构造函数。

原则上讲,你也可以通过如下声明来支持\texttt{Multi<>}进行赋值操作:
\begin{lstlisting}
    template<typename... Types>
    class Multi : private Base<Types>...
    {
        ...
        // 派生所有赋值运算符
        using Base<Types>::operator=...;
    };
\end{lstlisting}

\section{后记}
逗号分隔的using声明列表由Robert Haberlach在\url{https://wg21.link/p0195r0}中首次提出。
最终被接受的是Robert Haberlach和Richard Smith发表于\url{https://wg21.link/p0195r2}的提案。

关于继承构造函数有一些核心的问题。最终修复这些问题的提案由Richard Smith发表于
\url{https://wg21.link/n4429}。

还有一个由Vicente J.Botet Escriba提出的提案。
除了lambda之外,它还支持重载普通函数、成员函数来实现泛型的\texttt{overload}函数。
然而,这个提议并没有进入C++17标准。详情请见\url{https://wg21.link/p0051r1}。


    \part{新的标准库组件}\label{part3}
    这一部分介绍C++17中新的标准库组件。

    \section{\texttt{std::byte}}\label{ch18}
通过\texttt{std::byte},C++17引入了一个类型来代表内存的最小单位:字节。
\texttt{std::byte}本质上代表一个字节的值,但并不能进行数字或字符的操作,
也不对每一位进行解释。对于不需要数字计算和字符序列的场景,这样会更加类型安全。

然而你,注意\texttt{std::byte}实现的大小和\texttt{unsigned char}一样,这意味着
它并不保证是8位,可能会更多。

\subsection{使用\texttt{std::byte}}
下面的代码展示了\texttt{std::byte}的核心能力:
\begin{lstlisting}
    #include <cstddef>  // for std::byte

    std::byte b1{0x3F};
    std::byte b2{0b1111'0000};

    std::byte b3[4] {b1, b2, std::byte{1}}; // 4个字节(最后一个是0)

    if (b1 == b3[0]) {
        b1 <<= 1;
    }

    std::cout << std::to_integer<int>(b1) << '\n';  // 输出:126
\end{lstlisting}
这里,我们定义了两个初始值不同的字节。\texttt{b2}的初始化使用了两个C++14引入的特性:
\begin{itemize}[leftmargin=*]
    \item 前缀\texttt{0b}允许定义二进制字面量
    \item \emph{数字分隔符}\texttt{'}可以增强数字字面量的可读性
    (它可以被放置在数字字面量中任意两个数字之间)。
\end{itemize}
注意列表初始化(使用花括号初始化)是唯一可以直接初始化\texttt{std::byte}对象的方法。
所有其他的形式都不能编译:
\begin{lstlisting}
    std::byte b1{42};       // OK(因为自从C++17起所有枚举都有固定的底层类型)
    std::byte b2(42);       // ERROR
    std::byte b3 = 42;      // ERROR
    std::byte b4 = {42};    // ERROR
\end{lstlisting}
这是将\texttt{std::byte}实现为枚举类型的一个直接后果。
花括号初始化使用了新的\hyperref[ch8.3]{用整数值初始化有作用域的枚举}特性。

这里不能使用隐式类型转换,这意味着你必须显式对整数值进行转换才能初始化字节数组:
\begin{lstlisting}
    std::byte b5[] {1};             // ERROR
    std::byte b6[] {std::byte{1}};  // OK
\end{lstlisting}
如果没有初始化,\texttt{std::byte}将是栈上的值未定义的对象:
\begin{lstlisting}
    std::byte b;    // 值未定义
\end{lstlisting}
像通常一样(除了原子类型),你可以使用花括号强制初始化为每一位为0:
\begin{lstlisting}
    std::byte b{};  // 等价于b{0}
\end{lstlisting}
\texttt{std::to\_integer<>}允许你将\texttt{std::byte}对象转换为整数值(包括\texttt{bool}
和\texttt{char}类型)。如果没有转换,将不能使用输出运算符。注意因为这个转换函数是模板,
所以你需要使用带有\texttt{std::}的完整名称:
\begin{lstlisting}
    std::cout << b1;    // ERROR
    std::cout << to_integer<int>(b1);       // ERROR(ADL在这里不起作用)
    std::cout << std::to_integer<int>(b1);  // OK
\end{lstlisting}
也可以使用using声明(但请只在局部作用域中这么做):
\begin{lstlisting}
    using std::to_integer;
    ...
    std::cout << to_integer<int>(b1);       // OK
\end{lstlisting}
如果要将\texttt{std::byte}用作布尔值也需要这样的转换。例如:
\begin{lstlisting}
    if (b2) ...                         // ERROR
    if (b2 != std::byte{0}) ...         // OK
    if (to_integer<bool>(b2)) ...       // ERROR(ADL在这里不起作用)
    if (std::to_integer<bool>(b2)) ...  // OK
\end{lstlisting}
因为\texttt{std::byte}被实现为底层类型是\texttt{unsigned char}的枚举类型,
所以它的大小总是1:
\begin{lstlisting}
    std::cout << sizeof(b);             // 总是1
\end{lstlisting}
它的位数依赖于底层类型\texttt{unsigned char}的位数,
你可以通过标准数字限制来获取位数:
\begin{lstlisting}
    std::cout << std::numeric_limits<unsigned char>::digits; // std::byte的位数
\end{lstlisting}
这等价于:
\begin{lstlisting}
    std::cout <<
        std::numeric_limits<std::underlying_type_t<std::byte>>::digits;
\end{lstlisting}
大多数时候结果是8,但在有些平台上可能不是。

\subsection{\texttt{std::byte}类型和操作}
这一节详细描述\texttt{std::byte}类型和操作。

\subsubsection{\texttt{std::byte}类型}
在头文件\texttt{<cstddef>}中,C++标准库以如下方式定义了\texttt{std::byte}:
\begin{lstlisting}
    namespace std {
        enum class byte : unsigned char {
        };
    }
\end{lstlisting}
也就是说,\texttt{std::byte}不是别的,只是一个带有一些位运算符操作的有作用域的枚举类型:
\begin{lstlisting}
    namespace std {
        ...
        template<typename IntType>
        constexpr byte  operator<<  (byte  b, IntType shift) noexcept;
        template<typename IntType>
        constexpr byte& operator<<= (byte& b, IntType shift) noexcept;
        template<typename IntType>
        constexpr byte  operator>>  (byte  b, IntType shift) noexcept;
        template<typename IntType>
        constexpr byte& operator>>= (byte& b, IntType shift) noexcept;

        constexpr byte& operator|= (byte& l, byte r) noexcept;
        constexpr byte  operator|  (byte  l, byte r) noexcept;
        constexpr byte& operator&= (byte& l, byte r) noexcept;
        constexpr byte  operator&  (byte  l, byte r) noexcept;
        constexpr byte& operator^= (byte& l, byte r) noexcept;
        constexpr byte  operator^  (byte  l, byte r) noexcept;
        constexpr byte  operator~  (byte b) noexcept;

        template<typename IntType>
        constexpr IntType to_integer (byte b) noexcept;
    }
\end{lstlisting}

\subsubsection{\texttt{std::byte}操作}
表\hyperref[t18.1]{\texttt{std::byte}的操作}列出了\texttt{std::byte}的所有操作。
\begin{table}[ht]
    \begin{tabular}{l|p{0.5\textwidth}}
        \hline
        \textbf{操作} & \textbf{效果} \\
        \hline
        \emph{构造函数} & 创建一个字节对象(调用默认构造函数时值未定义) \\
        \emph{析构函数} & 销毁一个字节对象(什么也不做) \\
        \texttt{=} & 赋予新值 \\
        \texttt{==、!=、<、<=、>、>=} & 比较字节对象 \\
        \texttt{<<、>>、|、\&、\textasciicircum、\textasciitilde} & 二元位运算符    \\
        \texttt{<<=、>>=、|=、 \&=、\textasciicircum =} & 修改自身的位运算符 \\
        \texttt{to\_integer<T>()} & 把字节对象转换为整数类型\texttt{T} \\
        \texttt{sizeof()} & 返回1 \\
        \hline
    \end{tabular}
    \caption{\texttt{std::byte}的操作}
    \label{t18.1}
\end{table}

\subsubsection*{转换为整数类型}
用\texttt{to\_integer<>()}可以把\texttt{std::byte}转换为任意基本整数类型
(\texttt{bool}、字符类型或者整数类型)。这也是必须的,
例如为了将\texttt{std::byte}和整数值比较或者将它用作条件:
\begin{lstlisting}
    if (b2) ...                         // ERROR
    if (b2 != std::byte{0}) ...         // OK
    if (to_integer<bool>(b2)) ...       // ERROR(ADL在这里不生效)
    if (std::to_integer<bool>(b2)) ...  // OK
\end{lstlisting}
另一个例子使用它的例子是\hyperref[ch18.2.2.2]{\texttt{std::byte} I/O}。

\texttt{to\_integer<>()}使用\texttt{static\_cast}来把\texttt{unsigned char}
转换为目标类型。例如:
\begin{lstlisting}
    std::byte ff{0xFF};
    std::cout << std::to_integer<unsigned int>(ff); // 255
    std::cout << std::to_integer<int>(ff);          // 也是255
    std::cout << static_cast<int>(std::to_integer<signed char>(ff)); // -1
\end{lstlisting}

\subsubsection*{\texttt{std::byte}的I/O}\label{ch18.2.2.2}
\texttt{std::byte}没有定义输入和输出运算符,因此不得不把它转换为整数类型再进行I/O:
\begin{lstlisting}
    std::byte b;
    ...
    std::cout << std::to_integer<int>(b);   // 以十进制值打印出值
    std::cout << std::hex << std::to_integer<int>(b); // 以十六进制打印出值
\end{lstlisting}
通过使用\texttt{std::bitset<>},你可以以二进制输出值(一串位序列):
\begin{lstlisting}
    #include <cstddef>  // for std::byte
    #include <bitset>   // for std::bitset
    #include <limits>   // for std::numeric_limits

    std::byte b1{42};
    using ByteBitset =
        std::bitset<std::numeric_limits<unsigned char>::digits>;
    std::cout << ByteBitset{std::to_integer<unsigned>(b1)};
\end{lstlisting}
上例中using声明定义了一个位数和\texttt{std::byte}相同的bitset类型,
之后把字节对象转换为整数来初始化一个这种类型的对象,最后输出了该对象。
最后值42将会有如下输出(假设一个\texttt{char}是8位):
\begin{lstlisting}
    00101010
\end{lstlisting}
另外,你可以使用\texttt{std::underlying\_type\_t<std::byte>}代替\texttt{unsigned char},
这样using声明的目的将更明显。

你也可以使用这种方法把\texttt{std::byte}的二进制表示写入一个字符串:
\begin{lstlisting}
    std::string s = ByteBitset{std::to_integer<unsigned>(b1)}.to_string();
\end{lstlisting}
如果你已经有了一个字符序列,你也可以像下面这样使用\nameref{ch31.2.2}:
\footnote{感谢Daniel Krügler指出这一点。}
\begin{lstlisting}
    #include <charconv>
    #include <cstddef>

    std::byte b1{42};
    // 译者注:此处原文写的是
    // int value = 42;
    // 应是作者笔误

    char str[100];
    std::to_chars_result res =
        std::to_chars(str, str+99, std::to_integer<int>(b1), 2);
    *res.ptr = '\0';    // 确保最后有一个空字符结尾
\end{lstlisting}
注意这种形式将不会写入前导0,这意味着对于值42,最后的结果是(假设一个\texttt{char}有8位):
\begin{lstlisting}
    101010
    // 译者注:此处原文写的是
    // 1111110
    // 应是作者笔误
\end{lstlisting}
可以使用相似的方式进行输入:以整数、字符串或bitset类型读入并进行转换。
例如,你可以像下面这样实现读入字节对象的二进制表示的输入运算符:
\begin{lstlisting}
    std::istream& operator>> (std::istream& strm, std::byte& b)
    {
        // 读入一个bitset
        std::bitset<std::numeric_limits<unsigned char>::digits> bs;
        strm >> bs;
        // 如果没有失败就转换为std::byte
        if (!std::cin.fail()) {
            b = static_cast<std::byte>(bs.to_ulong());  // OK
        }
        return strm;
    }
\end{lstlisting}
注意我们必须使用\texttt{static\_cast<>()}来把bitset转换成的unsigned long转换为
\texttt{std::byte},列表初始化将不能工作,因为会发生窄化:
\footnote{使用gcc/g++时,如果没有编译选项\texttt{-pedantic-errors},
窄化初始化也可能通过编译。}
\begin{lstlisting}
    b = std::byte{bs.to_ulong()};   // ERROR:发生窄化
\end{lstlisting}
并且我们也没有其他的初始化方法了。

另外,你也可以使用\nameref{ch31.2.1}来从给定的字符序列读取:
\begin{lstlisting}
    #include <charconv>

    const char* str = "101001";
    int value;
    std::from_chars_result res =
        std::from_chars(str, str+6, // 要读取的字符范围
                        value,      // 读取后存入的对象
                        2);         // 2进制
\end{lstlisting}

\subsection{后记}
Neil MacIntosh在发表于\url{https://wg21.link/p0298r0}的提案中首次提出\texttt{std::byte}。
最终被接受的提案由Neil MacIntosh发表于\url{https://wg21.link/p0298r3}。


    \section{文件系统库}\label{ch20}

\subsection{基本的例子}

\subsubsection{打印文件系统路径类的属性}

\subsubsection{用\texttt{switch}语句处理不同的文件系统类型}

\subsubsection{创建不同类型的文件}

\subsubsection{使用并行算法处理文件系统}

\subsection{原则和术语}

\subsubsection{通用的可移植性路径分隔符}

\subsubsection{命名空间}

\subsubsection{文件系统路径}\label{ch20.2.3}
    \chapter{类型trait扩展}\label{ch21}

\section{类型trait后缀\texttt{\_v}}\label{ch21.1}

\section{新的类型trait}
\subsubsection{类型trait \texttt{is\_aggregate<>}}\label{ch21.2.1}


    \part{已有标准库的拓展和修改}\label{part4}
    这一部分介绍C++17对已有标准库组件的拓展和修改

    \chapter{子串和子序列搜索器}\label{ch24}
\section{使用子串搜索器}
\subsection{通过\texttt{search()}使用搜索器}
\subsection{直接使用搜索器}\label{ch24.1.2}
    \chapter{其他工具函数和算法}\label{ch25}

\section{\texttt{size()},\texttt{empty()},\texttt{data()}}

\subsection{泛型\texttt{size()}函数}

\subsection{泛型\texttt{empty()}函数}

\subsection{泛型\texttt{data()}函数}

\section{\texttt{as\_const()}}

\subsection{以常量引用捕获}\label{ch25.2.1}


    \part{专家的工具}\label{part5}
    这一部分介绍了普通应用程序员通常不需要知道的新的语言特性和库。
    它主要包括了为编写基础库和语言特性的程序员准备的用来解决特殊问题语言特性(例如修改了堆内存的管理方式)。

    \chapter{使用\texttt{new}和\texttt{delete}管理超对齐数据}\label{ch30}

\section{使用带有对齐的\texttt{new}运算符}

\subsection{不同的动态/堆内存竞争}

\subsection{使用\texttt{new}表达式传递对齐}

\subsubsection{使用\texttt{new}传递对齐会影响\texttt{delete}}
\subsubsection{实现放置\texttt{delete}}\label{ch30.1.2.2}

\section{实现内存对齐分配的\texttt{new()}运算符}
\subsection{在C++17之前实现对齐的内存分配}
\subsection{实现类型特化的\texttt{new()}运算符}\label{ch30.2.2}

\section{实现全局的\texttt{new()}运算符}

\section{追踪所有\texttt{::new}调用}\label{ch30.4}

\section{后记}


    \part{一些通用的提示}\label{part6}
    这一部分介绍了一些有关C++17的通用的提示,例如对C语言和废弃特性的兼容性更改。

\end{document}